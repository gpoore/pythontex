\documentclass[11pt]{article}

% Engine-specific settings
% pdftex:
\ifcsname pdfmatch\endcsname
    \usepackage[T1]{fontenc}
    \usepackage[utf8]{inputenc}
\fi
% xetex:
\ifcsname XeTeXinterchartoks\endcsname
    \usepackage{fontspec}
    \defaultfontfeatures{Ligatures=TeX}
\fi
% luatex:
\ifcsname directlua\endcsname
    \usepackage{fontspec}
\fi
% End engine-specific settings

\usepackage{lmodern}
\usepackage{amssymb,amsmath}
\usepackage{graphicx}
\usepackage{fullpage}
\usepackage[keeptemps=all, makestderr, usefamily={rust, rs}]{pythontex}

\begin{document}


\section*{Rust (\texttt{rust})}

Inline:  \rust{format!("$3 + 5 = {}$", 3+5)}.  \rusts{$4 + 6 = !{4+6}$}.


\begin{rustcode}
println!("Hello from Rust!");
println!("Running command family ``{}'', session ``{}'', restart ``{}''.", rstex.family, rstex.session, rstex.restart);
\end{rustcode}



\begin{rustblock}
println!("{}\\endinput", 2.0_f64.powf(8.0_f64));
\end{rustblock}

Printed output:  \printpythontex.

\begin{rustsub}
\color{blue}
\begin{Verbatim}
2.0_f64.powf(8.0_f64) = !{2.0_f64.powf(8.0_f64)}
\end{Verbatim}
\end{rustsub}


\section*{Rust (\texttt{rs})}

Inline:  \rs{format!("$3 + 5 = {}$", 3+5)}.  \rss{$4 + 6 = !{4+6}$}.


\begin{rscode}
println!("Hello from Rust!");
println!("Running command family ``{}'', session ``{}'', restart ``{}''.", rstex.family, rstex.session, rstex.restart);
\end{rscode}



\begin{rsblock}
println!("{}\\endinput", 2.0_f64.powf(8.0_f64));
\end{rsblock}

Printed output:  \printpythontex.

\begin{rssub}
\color{blue}
\begin{Verbatim}
2.0_f64.powf(8.0_f64) = !{2.0_f64.powf(8.0_f64)}
\end{Verbatim}
\end{rssub}



\section*{Errors}

\begin{rustblock}[error]
println!("{}\\endinput", 2.0_f64.powf(8.0_f64);
\end{rustblock}

\stderrpythontex


\end{document}
