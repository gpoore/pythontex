\documentclass[11pt]{article}

% Engine-specific settings
% pdftex:
\ifcsname pdfmatch\endcsname
    \usepackage[T1]{fontenc}
    \usepackage[utf8]{inputenc}
\fi
% xetex:
\ifcsname XeTeXinterchartoks\endcsname
    \usepackage{fontspec}
    \defaultfontfeatures{Ligatures=TeX}
\fi
% luatex:
\ifcsname directlua\endcsname
    \usepackage{fontspec}
\fi
% End engine-specific settings

\usepackage{lmodern}
\usepackage{amssymb,amsmath}
\usepackage{graphicx}
\usepackage{fullpage}
\usepackage[keeptemps=all, makestderr, usefamily=maxima]{pythontex}

\begin{document}

\section*{Maxima}

Inline:  \maxima{factor(x^2-x-6);}

Code environment:
\begin{maximacode}
a:'integrate(x^(n-1)*exp(-x),x)$
a=ev(a,nouns);
\end{maximacode}

Block environment:
\begin{maximablock}
solve(x^2+1,x);
\end{maximablock}
Printed output:
\printpythontex

Verbatim:
\begin{maximaverbatim}
solve(x^2-2,x);
\end{maximaverbatim}

Sub environment:
\begin{maximasub}
Some text \textcolor{blue}{!{x=743}} and then more text.
\end{maximasub}

Again, with command:  \maximas{Some text \textcolor{red}{!{phi=3^2-\%pi}} and then more text.}

\end{document}
